\chapter{Betriebsschwingungsanalyse - HKA}
\label{sec: Hauptkapitel 3}

\section{Aufgabenstellung}
    % Grundziel
    %----------
    Auch bei dieser Laboraufgabe soll eine Betriebsschwingungsanalyse
    durchgeführt werden. Diesmal ist das Zeitsignal jedoch an mehreren, vorher
    definierten, Punkten an der Tragfläche aufzunehmen. Für das Schleppen des
    Motors kann der Versuchsaufbau der vorhergehenden Laborübung verwendet
    werden.
    \\
    %----------------------------------------------------------------------------

    % Ausdetaillierung
    %-----------------
    \noindent
    Nach Aufnahme der Zeitsignale soll eine Hauptkomponentenanalyse (HKA)
    durchgeführt werden. Dies soll einmal für das gesamte Signal gemacht werden.
    In einem 2. Schritt soll die Rampe in 10 gleich große Intervalle unterteilt
    werden. Zu jedem dieser Zeitintervalle gilt es, anschließend eine HKA
    durchzuführen.
%================================================================================

\section{Versuchsaufbau}
%================================================================================

\section{Ergebnisse}