\chapter{Messaufbau}
\label{sec: Hauptkapitel 2}
%
\section{Versuchsaufbau}
%
Um die Schüttgutdichte zu ermitteln müssen die diversen Schüttgute zuerst gewogen werden. Hierfür wird eine handelsübliche Kaffeetasse
ohne Füllung gewogen um ihr Eigengewicht festzustellen. Danach wird sie bis zum Rand mit den zu wiegenden Materialien befüllt. 
Um das Volumen der Tasse in jedem Versuch gleich weit auszufüllen, wird die Tasse über den Rand abgestreift. Dadurch 
wird das Volumen optimal genutzt. Die befüllte Tasse wird auf einer Waage gewogen und danach wieder entleert. Jedes Schüttgut 
wird auf diese Weise mehrmals gemessen. Die gemessenen Werte werden in ein Excel-File übertragen.
%
\section{Messungen Stahlkugeln}
%
Zuerst werden drei Tassen mit Stahlkugeln befüllt und gewogen. Danach werden die Stahlkugeln wieder in deren Aufbewahrungsbehälter 
zurück geleert. Anschließend werden die Tassen wieder befüllt und erneut gewogen, dieser Vorgang wird insgesamt dreimal wiederholt.
Durch diese Messung wird für jede Tasse ein mittleres Gewicht der Kugeln bestimmt. Aus dem Gewicht der Kugeln und dem Volumen der 
Tasse wird so die Schüttgutdichte bestimmt. Aus dem Gewicht der Kugeln und dem Gewicht einer einzelnen Kugel wird berechnet, 
wie viele Kugeln in der Tasse sind. Dadurch wird aus dem Gesamtvolumen aller Kugeln und dem Volumen der Tasse die Packungsdichte 
berechnet. Mehr dazu in Kapitel \ref{sec: Hauptkapitel 3}.
%
\section{Messungen Kaffee}
%
Um die Schüttgutdichte von Kaffee zu bestimmen, werden Kaffeepulver und Kaffeebohnen in Tassen gewogen. Die Masse der Tassen 
wurde bereits durch die Messung der Stahlkugeln bestimmt. Es werden mehrere Schüttgutdichten bestimmt, darunter Kaffeebohnen 
in trockener Form, in feuchter Form mit $20\%$ Feuchtigkeit sowie mit Verunreinigung. Ebenfalls wird Kaffeepulver in trockener Form, 
in feuchter Form mit $20\%$ Feuchtigkeit sowie mit Verunreinigung gemessen. Die Verunreinigung in den Kaffeebohnen besteht aus einem Masseanteil 
von $20\%$ Kaffeepulver und die Verunreinigung im Kaffeepulver besteht aus einem Masseanteil von $20\%$ Kaffeebohnen. 
Die beiden verunreinigten Schüttgute werden ebenfalls mit einem Feuchtigkeitsanteil von $20\%$ gemessen. Alle Messungen 
werden drei mal durchgeführt und in eine Excelliste eingetragen. Mithilfe der Excelliste werden die Packungsdichten berechnet 
und signifikante Unterschiede über Normalverteilungen bestimmt. Eine Beschreibung und Diagramme hierzu folgen in 
Kapitel \ref{sec: Hauptkapitel 3}.


