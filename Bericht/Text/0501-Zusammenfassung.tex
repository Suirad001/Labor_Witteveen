\chapter{Zusammenfassung/Schlussfolgerung}
\label{sec: Zusammenfassung}

% Zielsetzung
%------------
\noindent
Ziel des Labors war es, den Einfluss von verschiedenen Parametern (Korngröße,
Verschmutzung, Feuchte) auf die Schüttgutdichte zu bestimmen.
Nach den Experimenten sollte beurteilt werden können, welche Parameter größeren und
welchen kleineren Einfluss auf die Schüttgutdichte haben.
\\
%------------------------------------------------------------------------------------

% Lösungsansätze
%---------------
\noindent
Um Aussagen über den Parametereinfluss treffen zu können, wurden einige Versuche
durchgeführt. Dabei wurde eine vollfaktorielle Versuchsanordnung gewählt. Als
Schüttgut wurde Kaffee verwendet.
Die Korngröße wurde modelliert, indem Kaffeepulver und Kaffebohnen verwendet wurden.
Die Feuchte durch Hinzugabe von Wasser zum Schüttgut simuliert und die Verschmutzung
durch Hinzugabe von Bohnen zum Pulver und umgekehrt.
\\
%------------------------------------------------------------------------------------

% Schlussfolgerungen
%-------------------
\noindent
Grob können die Ergegbnisse in 3 verschiedene Kategorien eingeteilt werden:
	\begin{itemize}
		\item Parameter und Wechselwirkungen mit hochsignifikantem Einfluss
		\item Parameter und Wechselwirkungen mit signifikantem Einfluss
		\item Parameter und Wechselwirkungen mit keinem signifikanten Einfluss
	\end{itemize}

% hochsignifikant
\noindent
Es hat sich herausgestellt, dass die Verunreinigung sich hochsignifikant auf die
Packungsdichte auswirkt. Außerdem liegt eine hochsignifikante 3-fach Wechselwirkung
zwischen Korngröße, Feuchtigkeit und Verunreinigung vor.
\\

% signifikant
\noindent
Die Feuchtigkeit hat einen signifikanten Einfluss auf die Packungsdichte und die
Wechselwirkungen zwischen Korngröße und Feuchtigkeit und zwischen Korngröße und
Verunreinigung sind ebenfalls signifikant.
\\

% nicht signifikant
\noindent
Als nicht signifikanter Einflussfaktor wurde die Korngröße identifiziert.
Auch die Wechselwirkung zwischen Feuchtigkeit und Verunreinigung 
ist deutlich nicht signifikant.
