\chapter{Einleitung}
\label{sec: Einleitung}

\section{Aufgabenstellung}
    % Allgemeine Aufgabenstellung
    %----------------------------
    Im Zuge dieser Laborübung soll ein Modellflugzeug schwingungstechnisch
    untersucht werden. Dabei gilt es, 4 verschiedene Aufgabenstellungen
    abzuhandeln. Um die Leserlichkeit zu erleichtern, werden die einzelnen
    Aufgabenstellung in eigenen Unterkapiteln jeweils vollständig abgearbeitet.
%================================================================================

\section{Vorbereitung}
    % Aufbereitung des Modellflugzeugs
    %---------------------------------
    Da für die folgenden Messungen oftmals lineare Zusammenhänge angenommen
    werden, sollen im vermessenen System möglichst keine Nichtlinearitäten
    vorkommen. Deshalb werden lose Flugzeugteile (z.B. lose hängende Ketten,
    wackelnde Finnen, ...) mit Klebeband fixiert.
    Konkret wurden beim verwendeten Modellflugzeug folgende Teile befestigt:
    %
    \begin{itemize}
        \item eine lose hängende Kette vorne beim Motor
        \item die Mittelfinne am Heck
        \item das Fahrwerk am Heck
    \end{itemize}
