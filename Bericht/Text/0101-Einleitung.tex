\chapter{Aufgabenstellung}
\label{sec: Einleitung}

% Allgemeine Aufgabenstellung
%----------------------------
Im Zuge dieser Laborübung soll selbstständig die Schüttgutdichte verschiedener Schüttgute berechnet werden.
Es werden zwei Versuche durchgeführt. Im ersten Versuch werden Stahlkugel verwendet. 
Als Schüttgute werden im zweiten Versuch Kaffee sowohl in Bohnenform als auch in gemahlener Form verwendet. 
Um Unterschiede in den einzelenen Schüttguten festzustellen, werden Bohen in feuchter und trockener Form, 
Pulver in feuchter und trockener Form sowie Bohnen und Pulver mit Verunreinigung verwendet. Als Verunreinigung 
wird für Bohnen Pulver und für Pulver Bohnen genutzt. Die verunreinigten Schüttgute werden ebenfalls in feuchter Form 
für die Berechnungen genutzt.
\\
