\chapter{Theorie}
\label{sec: Hauptkapitel 1}

\section{Design of Experiment (DOE)}
	% Einleitung
	%-----------
	\noindent
	DOE beschäftigt sich mit der strukturierten Analyse von Problemstellungen.
	Ziel ist es, Zusammenhänge zwischen einzelnen Größen zu erkennen und zu
	beschreiben. Dabei gibt es verschiedene Versuchsstrategien \cite{ref:DOE_IL}.
	Da im Labor dabei die vollfaktorielle Versuchsplanung verwendet wird, wird
	diese hier näher beleuchtet.
	\\

	% Vollfaktorielle Versuchsplanung
	%--------------------------------
	\noindent
	Bei der vollfaktoriellen Versuchsplanung werden alle zu messenden Parameter
	gemessen und somit der ganze Versuchsraum aufgespannt. Dies hat den Vorteil,
	dass sich alle Haupteffekte und Wechselwirkungen bestimmen lassen und sich die
	größtmögliche Informationstiefe ergibt.
	Nachteilig hingegen ist, dass die Anzahl der durchzuführenden Versuche sehr
	schnell sehr groß wird. Bezeichnet $s$ die Anzahl der Stufen und $k$ die Anzahl
	der zu variierenden Faktoren, so wächst die Anzahl der Versuchsanordnungen mit:
	%
	\begin{equation}
		m = s^k
		\label{eq: Versuchsanordnungen}
	\end{equation}
	%
	Werden nun aus statistischen Gründen noch mehrere Messungen $n$ je
	Versuchsanordnung durchgeführt, ergibt sich die Anzahl der Versuche zu
	%
	\begin{equation}
		N = m \cdot n = s^k \cdot n
		\label{eq: Versuchsanzahl}
	\end{equation}
%===========================================================================================

\section{Schüttguttheorie \cite{ref:Stat_Versuchsplanung}}
	% Schüttdichte
	%-------------
	\noindent
	Bei der Auslegung von Förderanlagen spielt das Schüttgutverhalten eine wichtige
	Rolle. Eine zentrale Größe bei der Charaktisierung des Schüttgutverhaltens ist
	dabei die Schüttdichte $\rho$. Diese ist nicht konstant sondern hängt von der
	Korngrößenverteilung, der Kornform, der Feuchtigkeit und der Verfestigung ab.
	Die Schüttdichte berechnet sich über
	%
	\begin{equation}
		\rho = \frac{m_G}{V}
		\label{eq: Schüttdichte}
	\end{equation}
	%
	\noindent
	wobei $m_G$ die Gutmasse bezeichnet und $V$ das Gesamtvolumen. Sie hat somit
	die Einheit
	$\left[ \frac{kg}{m^3} \right]$.
	\\
	%----------------------------------------------------------------------------------------

	% Packungsdichte
	%---------------
	\noindent
	Eine weitere, wichtige Größe stellt die Packungsdichte $\delta$ dar. Diese ist
	definiert als das Verhältnis der Summe der einzelnen Volumina der
	Schüttgutelemente $V_o$ zum umschließenden Volumen $V_G$ (siehe Formel
	\ref{eq: Packungsdichte}).
	%
	\begin{equation}
		\delta = \frac{V_o}{V_G}
		\label{eq: Packungsdichte}
	\end{equation}
	%----------------------------------------------------------------------------------------

	% Zus. Packungsdichte und Schüttdichte
	%-------------------------------------
	\noindent
	Die Packungsdichte hängt auch mit der Schüttdichte zusammen. Die Gesamtmasse
	setzt sich aus der Masse des Schüttguts und der Masse der Luft in den
	Zwischenräumen zwischen den einzelnen Elementen zusammen (siehe Formel
	\ref{eq: Gesamtmasse})
	%
	\begin{equation}
		m_G = m_o + m_L
		\label{eq: Gesamtmasse}
	\end{equation}
	%
	\noindent
	Wird nun berücksichtigt, dass sich die Masse im Allgemeinen zu $\rho \cdot V$
	ergibt, und in Formel \ref{eq: Schüttdichte} eingesetzt, ergibt sich folgender
	Zusammenhang für die Schüttdichte:
	%
	\begin{equation}
		\rho = \frac{V_o \cdot \rho_o}{V} + \frac{V_L \cdot \rho_L}{V}
		\label{eq: Schüttdichte_lang}
	\end{equation}
	%
	\noindent
	Weil das Luftvolumen $V_L$ sehr klein ist, kann der rechte Term in Gleichung
	\ref{eq: Schüttdichte_lang} vernachlässigt werden und die Gleichung vereinfacht
	sich zu
	%
	\begin{equation}
		\rho \approx \frac{V_o \cdot \rho_o}{V} = \delta \cdot \rho_o
		\label{eq: Zus_Schüttdichte_Packungsdichte}
	\end{equation}
